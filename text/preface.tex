\chapter*{Introduction}
\addcontentsline{toc}{chapter}{Introduction}

Despite the world's effort to eradicate debilitating human diseases, blindness is still a common condition that worsens the quality of people's lives. A recent study showed that in 2015, there were an estimated 36 million blind people \citep{ackland2017world}. Although cure has been found for some diseases, some causes of blindness are still incurable with irreversible consequences. Additionally, injuries to the early visual pathway also occur, and they need to be treated.

% In recent years, a relatively new technology emerged - visual prostheses. 
Among many other efforts to restore sight in totally blind people, one of the possible solutions is a relatively new but promising technology of brain implants \citep{kilgore_2015}. Visual prostheses, some of which are currently being clinically tested \citep{fernandez2021visual}, are usually implanted in the retina, LGN, or visual cortex (V1). Using direct electrical \citep{cogan2008neural} or optogenetic \citep{edward2018towards} stimulation of neurons, they evoke visual percepts enabling blind people to see again.

Current visual implants are not only hardly available, but they are also not providing a high-resolution sight yet. However, better stimulation techniques and stimulation protocols are being developed. In this work, the latter will be the subject of our attention.

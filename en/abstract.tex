%%% A template for a simple PDF/A file like a stand-alone abstract of the thesis.

\documentclass[12pt]{report}

\usepackage[a4paper, hmargin=1in, vmargin=1in]{geometry}
\usepackage[a-2u]{pdfx}
\usepackage[utf8]{inputenc}
\usepackage[T1]{fontenc}
\usepackage{lmodern}
\usepackage{textcomp}

\begin{document}

Neighboring neurons in the primary visual cortex (V1), the first cortical area processing visual information, are selective to stimuli presented in neighboring positions of the visual field with a specific edge orientation. In this way, they form the so-called retinotopic and orientation maps of V1. Due to the absence of high-resolution cortical stimulation devices, vision restoration through prosthetic implants in V1 has not yet taken advantage of the orientation maps. However, the availability of cortical implants with stimulation resolution high enough to target separate orientation columns can be anticipated soon. 

Since other stimulus features are also encoded in the cortex, such as color, size, or phase, but cannot be reliably engaged even by high-resolution stimulation, in this thesis, we ask the question of how well can visual stimuli be encoded in V1 if only orientation and position preference is known. To address this question, we propose a deep neural network (DNN) providing a scalar neural activity descriptor for any targeted cortical location and multiple different orientations. This is achieved by employing a rotation-equivariant convolutional neural network (reCNN) with the last layer having only one channel for each orientation, returning the desired three-dimensional feature tensor. A specialized readout estimates the neurons’ positions and their orientation preference, using them to yield one value per neuron from the core’s last layer, from which the neural response is predicted. These scalar features might serve as an input for the future stimulation protocol.

The proposed network was trained both on an experimental and a synthetic dataset. The synthetic dataset generated by Antolík et al. model of a cat’s V1 provided locations of neurons along with their orientation preference. This was used to examine the DNN model’s precision in both neural response prediction and neural location and preferred orientation estimates. 

This work proposes a DNN model yielding a 0.8327 correlation fraction with respect to the control model. Moreover, with the proposed reCNN model achieving sufficient performance, we were able to reconstruct the in-silico orientation maps. This promising result suggests that targeting cortical columns of neurons matching in orientation and position preference might improve information encoding via prosthetic stimulation.


\end{document}

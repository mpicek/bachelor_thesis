
%%% A template for a simple PDF/A file like a stand-alone abstract of the thesis.

\documentclass[12pt]{report}

\usepackage[a4paper, hmargin=1in, vmargin=1in]{geometry}
\usepackage[a-2u]{pdfx}
\usepackage[utf8]{inputenc}
\usepackage[T1]{fontenc}
\usepackage{lmodern}
\usepackage{textcomp}

\begin{document}


Sousední neurony v primární zrakové kůře (V1), první kortikální oblasti zpracovávající vizuální informace, jsou selektivní vůči podnětům prezentovaným v sousedních polohách zorného pole se specifickou hranovou orientací. Tvoří tak tzv. retinotopické a orientační mapy V1. Vzhledem k absenci zařízení pro kortikální stimulaci s vysokým rozlišením zatím obnovení zraku pomocí protetických implantátů ve V1 nevyužilo výhody orientačních map. Brzy však lze očekávat dostupnost kortikálních implantátů s dostatečně vysokým rozlišením stimulace, aby bylo možné zacílit na samostatné orientační sloupce.

Vzhledem k tomu, že v kortexu jsou zakódovány i jiné stimulační prvky, jako je barva, velikost nebo fáze, ale nelze je spolehlivě zapojit ani stimulací s vysokým rozlišením, klademe si v této práci otázku, jak dobře lze vizuální stimuly zakódovat ve V1 pokud je známa pouze orientace a preference polohy. K vyřešení této otázky navrhujeme hlubokou neuronovou síť (DNN) poskytující deskriptor nervové aktivity pro jakoukoli cílovou kortikální pozici a danou orientační preferenci. Toho je dosaženo použitím rotačně-ekvivariantní konvoluční neuronové sítě (reCNN) s poslední vrstvou, která má pouze jeden kanál pro každou orientaci, vracející požadovaný trojrozměrný tenzor hodnot. Specializovaný výstup odhaduje polohy neuronů a jejich preferenci orientace a používá je k získání jedné hodnoty pro daný neuron z poslední vrstvy jádra, ze které se předpovídá neurální aktivita. Tyto skalární hodnoty mohou sloužit jako vstup pro budoucí stimulační protokol.

Navrhovaná síť byla trénována jak na experimentálním, tak na syntetickém datasetu. Syntetický dataset vytvořený Antolíkem a jeho kolegy pomocí modelu kočičí V1 poskytuje informace o umístění neuronů spolu s jejich preferovanou orientací. To bylo použito k výzkumu přesnosti modelu DNN v predikci neurální odezvy a kortikální poloze a odhadu preferované orientace.

Tato práce navrhuje model DNN poskytující korelační zlomek 0,8327 vzhledem ke kontrolnímu modelu. Navíc, jelikož navrhovaný model reCNN dosáhl dostatečné přesnosti, byli jsme schopni rekonstruovat orientační mapy in-silico. Tento slibný výsledek naznačuje, že zacílení kortikálních sloupců neuronů odpovídajících orientaci a preferenci polohy by mohlo zlepšit kódování informací prostřednictvím protetické stimulace.


\end{document}


